\begin{titlepage}
 \setlength{\unitlength}{\textwidth}
  \begin{picture}(1,1.41)              % pagxforma
    \thinlines
    \put(0.05,0.05){\line(0,1){1.31}}         % vertikala linio maldekstra
    \put(0.95,0.05){\line(0,1){1.31}}         % vertikala linio dekstra
    \put(0.05,0.05){\line(1,0){0.9}}            % horizontala linio malsupra
    \put(0.05,1.36){\line(1,0){0.9}}         % horizontala linio supra
    \thicklines
    \put(0,0){\line(0,1){1.41}}         % vertikala linio maldekstra
    \put(1,0){\line(0,1){1.41}}         % vertikala linio dekstra
    \put(0,0){\line(1,0){1}}            % horizontala linio malsupra
    \put(0,1.41){\line(1,0){1}}         % horizontala linio supra
    \put(0.5,1.1){   \makebox(0,0){\huge Multatuli}}
    \put(0.3,1.0){\line(1,0){0.4}}
    \put(0.5,0.9){ \makebox(0,0){\Huge Saidjah kaj Adinda}    }
  \end{picture}
\end{titlepage}
\pagestyle{empty}
% \vspace*{\textheight}
\hbox{}
\vfill
\begin{minipage}[t]{\textwidth}
`Sa\"{\i}djah en Adinda' estas fama rakonto kiu aperis kadrita
en la romano `Max Havelaar', kiun skribis la nederlanda a\u{u}toro
Multatuli. Tiu libro  unuafoje estis publikigita en la jaro 1860.

Multaj malajaj esprimoj estas uzataj, kiuj estas klarigataj per
kelkaj notoj. Mi elektis multajn ne traduki \^car
anka\u{u} tiel estas en la nederlanda originalo.

La plej nova versio de tiu \^ci traduko \^ciam trovi\^gas je:\\
\href{http://purl.org/NET/mihxil/saidjah/}{http://purl.org/NET/mihxil/saidjah/}\\

Michiel  Meeuwissen $<$mihxil@gmail.com$>$\\

Kompostita per \LaTeX\\
Versio: \input{revisio}
\end{minipage}
\newpage
\pagestyle{plain}
\setcounter{page}{1}
